\documentclass[a4paper,12pt]{article}
\usepackage{graphicx}
\usepackage{amsmath}
\usepackage{mathtools}
\usepackage[cm]{fullpage}
\newcommand{\mname}{My name is: Currell Berry}
\begin{document}


\section*{\centering Math 2403 Formula Sheet}
\section{Chapter 2}

\begin{flalign*}% left aligned
\cos\theta_1 \cos\theta_2-\sin\theta_1\sin\theta_2 &= \cos(\theta_1 +\theta_2) &\\
\sin\theta_1 \cos\theta_2 + \cos\theta_1 \sin\theta_2 &= \sin(\theta_1+\theta_2) &% Need tailing alignment char to get all the way left
\end{flalign*}


\verb!book!
The first five International Congresses of Mathematicians
were held in the following cities:
\begin{quote}
\begin{tabular}{lll}
Chicago&U.S.A.&1893\\
Z\"{u}rich&Switzerland&1897\\
Paris&France&1900\\
Heidelberg&Germany&1904\\
Rome&Italy&1908
\end{tabular}
\end{quote}

\begin{description}
\item[General FOLDE:] \(\frac{dy}{dt}+p(t)y = g(t)\). \text{integrating factor} $u(t) = e^{p(t) \,dt}$

\end{description}
\begin{tabular}{c|c|c}
General FOLDE \(\frac{dy}{dt}+p(t)y = g(t)\).  integrating factor $u(t) = e^{p(t) \,dt}$ &
this is a long body of text & but not as long as this body of text \\
\end{tabular}
\begin{equation}
f(x) =sin(x)-e^{(-3x)}
\end{equation}
that is placed on a line by itself
\[
u(t) = e^{\int_0^\infty p(t) \,dt}
\]
\begin{figure}[h] 
\centering
\includegraphics[scale=1]{singraph} 
\caption{Graph of sin(x)}
\label{fig: awesome image}
\end{figure}
\end{document}
