%Currell Berry, 2/9/2016
%based on Steve Seiden's CS Cheatsheet

\def\letterheader{
\ifdefined\pdfpagewidth
\pdfpagewidth 8.5 true in
\pdfpageheight 11 true in
\fi
}

\font\tenbb=msbm10\relax
\font\eightrm=cmr8\relax
\font\bigger =cmr10 scaled\magstep2

\def\Natural{\hbox{\tenbb N}}
\def\Z{\hbox{\tenbb Z}}
\def\Real{\hbox{\tenbb R}}

\newcount\DisSmall
\newcount\DisCount
\newcount\DisNumber
\newdimen\DisSpace
\newdimen\DisLSpace
\newdimen\DisLeftSkip
\newdimen\DisParFillSkip
\newdimen\DisBaseLineSkip


\def\BeginTable#1{
\vfil
\begintable
\multicolumn#1\title\eltt
}
\def\Wider{
\def\stablesleft{\vskip 0em\hfil}
\def\stablesright{\hfil\vskip 0em}
}

\def\EndTable{
\endtable
\vfil
\eject
}

\def\Fm#1\Mf{\hskip 0pt plus 1fil\advance\DisCount by 1\hbox{\hskip\DisSpace\ifnum\DisNumber=1{\bf \number\DisCount. }\fi$\ifnum\DisSmall=0\displaystyle\fi#1$}}

\def\SNDis#1 #2pt{\MyDis{#1}{#2}{1}{20}{1}}
\def\NDis#1 #2pt{\MyDis{#1}{#2}{0}{29}{1}}
\def\SDis#1pt{\MyDis{1}{#1}{1}{20}{0}}
\def\Dis#1pt{\MyDis{1}{#1}{0}{29}{0}}

\def\Display#1{
\def\twlrm{}
\def\sixrm{}
\input #1 {}
\centerline{\box\graph}
}

\def\MyDis#1#2#3#4#5{%
\par\noindent%
\DisCount=#1%
\DisNumber=#5%
\advance\DisCount by -1%
\DisSmall=#3%
\DisSpace=#2 pt%
\DisLSpace=-#2 pt%
\DisLeftSkip=\leftskip%
\DisParFillSkip=\parfillskip%
\DisBaseLineSkip=\baselineskip%
\baselineskip=#4 pt%
\leftskip=\DisLSpace%
\parfillskip=0pt%
\vskip-#4 pt%
\par\noindent%
\break%
}

\def\EndDis{%
\par\noindent%
\vskip\belowdisplayskip%
\par\noindent%
\baselineskip=\DisBaseLineSkip%
\leftskip=\DisLeftSkip%
\parfillskip=0.0pt plus 1.0fil%
}

\def\frac#1 #2 {{#1 \over #2}}
\def\sfrac#1 #2 {\hbox{$#1 \over #2$}}
\def\Par{\par\vskip 3pt}
\def\center#1{{\hfil #1\hfil}}
\def\centertwo#1#2{\line {\hss #1\hss #2\hss}}
\def\centerthree#1#2#3{\line {\hss #1\hss #2\hss #3\hss}}
\def\sign{\hbox{\rm sign}}
\def\etc{$\ldots$}
\def\Ldots{,\ldots,}
\def\Hrule{\vskip 3pt\hrule\vskip 3pt}
\def\Bar{\vert}
\def\And{\wedge}
\def\Or{\vee}
\def\E{\mathop{\rm E}\nolimits}
\def\Pr{\mathop{\rm Pr}\nolimits}
\def\abs{\mathop{\rm abs}\nolimits}
\def\deg{\mathop{\rm deg}\nolimits}
\def\perm{\mathop{\rm perm}\nolimits}
\def\sinh{\mathop{\rm sinh}\nolimits}
\def\cosh{\mathop{\rm cosh}\nolimits}
\def\sech{\mathop{\rm sech}\nolimits}
\def\csch{\mathop{\rm csch}\nolimits}
\def\coth{\mathop{\rm coth}\nolimits}
\def\tanh{\mathop{\rm tanh}\nolimits}

\def\arccot{\mathop{\rm arccot}\nolimits}
\def\arcsec{\mathop{\rm arcsec}\nolimits}
\def\arccsc{\mathop{\rm arccsc}\nolimits}
\def\arcsinh{\mathop{\rm arcsinh}\nolimits}
\def\arccosh{\mathop{\rm arccosh}\nolimits}
\def\arctanh{\mathop{\rm arctanh}\nolimits}
\def\arccoth{\mathop{\rm arccoth}\nolimits}
\def\arcsech{\mathop{\rm arcsech}\nolimits}
\def\arccsch{\mathop{\rm arccsch}\nolimits}

\def\ramsey{\mathop{\rm r}\nolimits}
\def\Var{\mathop{\rm VAR}\nolimits}
\def\Subset#1#2{ \bigg\{{#1 \atop #2} \bigg\}}
%\def\subset#1#2{ \big\{{#1 \atop #2} \big\}}
\def\Cycle#1#2{ \bigg[{#1 \atop #2} \bigg]}
\def\cycle#1#2{ \big[{#1 \atop #2} \big]}
\def\Eul#1#2{ \bigg\langle{#1 \atop #2} \bigg\rangle}
\def\eul#1#2{ \big\langle{#1 \atop #2} \big\rangle}
\def\Euls#1#2{ \bigg\langle\!\!\!\bigg\langle{#1 \atop #2} \bigg\rangle\!\!\!\bigg\rangle}
\def\fivrm{}
\def\euls#1#2{ \big\langle\!\!\big\langle{#1 \atop #2} \big\rangle\!\!\big\rangle}

%CB additions
\def\compressedDisplay#1{\vskip -8pt\relax$$#1$$}

\def\bra#1{\left \langle #1 \right \vert}
\def\ket#1{\left \vert #1 \right \rangle}
\def\brakettwo#1#2{\left \langle #1\kern-2pt\mathbin\vert\kern-2pt #2  \right \rangle}
\def\braketthree#1#2#3{\left \langle #1\mathclose\vert#2\mathopen\vert#3\right \rangle}

\def\plainoperator#1{{\rm #1}\ }

\def\Dots{,\dots,}

\def\blackboard#1{\rm I\!#1}
%\def\blackboard#1{{\bf #1}}
\def\sloppy{\tolerance 9999 \emergencystretch 3em\relax}
\def\comment#1{}

%#1 is the box to use
%#2 is the height
\def\makeandprintfullheightcols#1#2{
%split the text twice to the dimension of #1 inches, 

\setbox40=\vsplit#1 to #2
\setbox60=\hbox to \dimen1{\vbox to #2{\unvbox40\vfill}}
\setbox41=\vsplit#1 to #2
\setbox61=\hbox to \dimen1{\vbox to #2{\unvbox41\vfill}}

\hbox to \hsize{\box60\hfil\box61}
%\medskip
%\hrule
%\vfill
\eject
}

\def\makeandprintthreefullheightcols#1#2{
%split the text thrice to the dimension of #2 inches, 

\setbox40=\vsplit#1 to #2
\setbox60=\hbox to \dimen1{\vbox to #2{\unvbox40\vfill}}
\setbox41=\vsplit#1 to #2
\setbox61=\hbox to \dimen1{\vbox to #2{\unvbox41\vfill}}
\setbox42=\vsplit#1 to #2
\setbox62=\hbox to \dimen1{\vbox to #2{\unvbox42\vfill}}

\hbox to \hsize{\box60\hfil\box61\hfil\box62\hfil}
%\medskip
%\hrule
%\vfill
\eject
}

\newdimen\remaindercolheight
%#1 is the box to use
\def\makeandprintequalheightcols#1{
%now put the remainder on the page with equal height splitting.

%do calculation for splitting two equal height columns
\remaindercolheight=\ht#1
\advance\remaindercolheight by \dp#1
\advance\remaindercolheight by \splittopskip
\divide\remaindercolheight by 2

\setbox40=\vsplit#1 to \remaindercolheight
\setbox60=\hbox to \dimen1{\vbox to \remaindercolheight{\unvbox40\vfill}}
\setbox41=\vsplit#1 to \remaindercolheight
\setbox61=\hbox to \dimen1{\vbox to \remaindercolheight{\unvbox41\vfill}}

\hbox to \hsize{\box60\hfil\box61}
}

%these come from http://insti.physics.sunysb.edu/~siegel/tex.shtml
\def\rgboo#1{\pdfliteral{#1 rg #1 RG}}
\def\rgbo#1#2{\rgboo{#1}#2\rgboo{0 0 0}}
\def\rgb#1#2{\mark{#1}\rgbo{#1}{#2}\mark{0 0 0}}
\def\pdfklink#1#2{%
	\noindent\pdfstartlink user
		{/Subtype /Link
		/Border [ 0 0 0 ]
		/A << /S /URI /URI (#2) >>}{\rgb{0 0 1}{#1}}%
\pdfendlink}

% URL support
%\def\URLcolor#1{\def\URLcolour{#1}}
%\URLcolor{blue}
%% Turn off the special meaning of ~ inside \URL{}.
%%\def\URL{\begingroup\catcode`\~=12\catcode`\_=12\relax\URL@}
%\def\URL #1{%
%%       \pdfannotlink user{   pdftex pre 0.14 ??
%        \pdfstartlink user{
%            /Subtype /Link
%            % w/o this you get an ugly box around the URL.
%            /Border [ 0 0 0 ]   % radius, radius, line thickness
%            /A <<
%                /Type /Action
%                /S /URI
%                /URI (#1)
%
%        }%
%        {\color{\URLcolour}\tt #1}%
%        \pdfendlink{}%
%%    \endgroup % Reset catcode of ~ and _
%}

\font\ninerm=cmr9 \font\eightrm=cmr8 \font\sixrm=cmr6
\font\ninei=cmmi9 \font\eighti=cmmi8 \font\sixi=cmmi6
\font\ninesy=cmsy9 \font\eightsy=cmsy8 \font\sixsy=cmsy6
\font\ninebf=cmbx9 \font\eightbf=cmbx8 \font\sixbf=cmbx6
\font\ninett=cmtt9 \font\eighttt=cmtt8
\font\nineit=cmti9 \font\eightit=cmti8
\font\ninesl=cmsl9 \font\eightsl=cmsl8

\newskip\ttglue
\def\tenpoint{\def\rm{\fam0\tenrm}% switch to 10-point type
\textfont0=\tenrm \scriptfont0=\sevenrm \scriptscriptfont0=\fiverm
\textfont1=\teni \scriptfont1=\seveni \scriptscriptfont1=\fivei
\textfont2=\tensy \scriptfont2=\sevensy \scriptscriptfont2=\fivesy
\textfont3=\tenex \scriptfont3=\tenex \scriptscriptfont3=\tenex
\textfont\itfam=\tenit \def\it{\fam\itfam\tenit}%
\textfont\slfam=\tensl \def\sl{\fam\slfam\tensl}%
\textfont\ttfam=\tentt \def\tt{\fam\ttfam\tentt}%
\textfont\bffam=\tenbf \scriptfont\bffam=\sevenbf
\scriptscriptfont\bffam=\fivebf \def\bf{\fam\bffam\tenbf}%
\tt \ttglue=.5em plus.25em minus.15em
\normalbaselineskip=12pt
\setbox\strutbox=\hbox{\vrule height8.5pt depth3.5pt width0pt}%
\let\sc=\eightrm \let\big=\tenbig \normalbaselines\rm}
\def\ninepoint{\def\rm{\fam0\ninerm}% switch to 9-point type
\textfont0=\ninerm \scriptfont0=\sixrm \scriptscriptfont0=\fiverm
\textfont1=\ninei \scriptfont1=\sixi \scriptscriptfont1=\fivei
\textfont2=\ninesy \scriptfont2=\sixsy \scriptscriptfont2=\fivesy
\textfont3=\tenex \scriptfont3=\tenex \scriptscriptfont3=\tenex
\textfont\itfam=\nineit \def\it{\fam\itfam\nineit}%
\textfont\slfam=\ninesl \def\sl{\fam\slfam\ninesl}%
\textfont\ttfam=\ninett \def\tt{\fam\ttfam\ninett}%
\textfont\bffam=\ninebf \scriptfont\bffam=\sixbf
\scriptscriptfont\bffam=\fivebf \def\bf{\fam\bffam\ninebf}%
\tt \ttglue=.5em plus.25em minus.15em
\normalbaselineskip=11pt
\setbox\strutbox=\hbox{\vrule height8pt depth3pt width0pt}%
\let\sc=\sevenrm \let\big=\ninebig \normalbaselines\rm}
\def\eightpoint{\def\rm{\fam0\eightrm}% switch to 8-point type
\textfont0=\eightrm \scriptfont0=\sixrm \scriptscriptfont0=\fiverm
\textfont1=\eighti \scriptfont1=\sixi \scriptscriptfont1=\fivei
\textfont2=\eightsy \scriptfont2=\sixsy \scriptscriptfont2=\fivesy
\textfont3=\tenex \scriptfont3=\tenex \scriptscriptfont3=\tenex
\textfont\itfam=\eightit \def\it{\fam\itfam\eightit}%
\textfont\slfam=\eightsl \def\sl{\fam\slfam\eightsl}%
\textfont\ttfam=\eighttt \def\tt{\fam\ttfam\eighttt}%
\textfont\bffam=\eightbf \scriptfont\bffam=\sixbf
\scriptscriptfont\bffam=\fivebf \def\bf{\fam\bffam\eightbf}%
\tt \ttglue=.5em plus.25em minus.15em
\normalbaselineskip=9pt
\setbox\strutbox=\hbox{\vrule height7pt depth2pt width0pt}%
\let\sc=\sixrm \let\big=\eightbig \normalbaselines\rm}
