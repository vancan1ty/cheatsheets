%TIKZ enables powerful plotting functionality
%\input tikz
%\usetikzlibrary{intersections,arrows}
%\input mbboard
%\input fontch
\input ../core/macros.tex

\input opmac
\input chelvet
\input tx-math
\input ctimes
\typosize[10/11.5]
%\usetimes
%\eightpoint
\font\biggest=phvb at 13 pt
\font\bigger=phvb at 12 pt
\font\smalltitle =phvr at 10pt
\font\smallfont=ptmr at 8pt

\tolerance=1000
\sloppy

\newif\iflong %makes control seqs \iflong, \longtrue, \longfalse
\longfalse
\def\piflong#1{\iflong#1
\else\fi}%
\long\def\lpiflong#1{\iflong#1
\else\fi}%

\def\xpair{{x_1,x_2}}

% the following two lines only work with pdftex.  set the paper size to letter, as pdftex
% defaults to A4 and this is not what we want here.
\pdfpagewidth 11 true in
\pdfpageheight 8.5 true in
\nopagenumbers

\def\complex{{\bf C}}
\def\real{{\bf R}}
\def\field{{\bf F}}
\def\bigJ{\bf \hat J^2}

\voffset=-0.80 in
\hoffset=-0.75 in
\hsize=10.6in
\vsize=8.05in
\parindent=0pt


%\splittopskip=18.3pt
\def\strutA#1#2{\vrule height#1 depth#2 width0pt}
\def\chapter{\vskip 5pt \bigger}
\def\hthree{\vskip 2pt \smalltitle}

\dimen1=3.40in
\newbox\bigbox


%#1: width of left col
%#2: width of right col
\def\dtablestart#1#2{\halign\bgroup \vtop{\parindent=0pt\hsize=#1\strut##\strut}\hfil&\hskip 0.10 in\vtop{\parindent=0pt\hsize=#2\strut##\strut}\cr}
\def\dtableend{\egroup}


\setbox\bigbox=\vbox {\hsize=\dimen1\strutA{\splittopskip}{0pt}{\biggest Currell Berry -- QM Study Sheet}
{\chapter Identities}

\vskip 0pt
{\hthree Spin 1/2 Identities:}
\vskip 3pt
\Dis 5pt 
\baselineskip=22pt
\Fm\ket{+x} = {1 \over \sqrt{2}}(\ket{+z}+\ket{-z})\Mf
\Fm\ket{-x} = {1 \over \sqrt{2}}(\ket{+z}-\ket{-z})\Mf
\Fm\ket{+y} = {1 \over \sqrt{2}}(\ket{+z}+i\ket{-z})\Mf
\Fm\ket{-y} = {1 \over \sqrt{2}}(\ket{+z}-i\ket{-z})\Mf
\EndDis
%% $$\ket{+x} = {1 \over \sqrt{2}}(\ket{+z}+\ket{-z}),\enskip
%% \ket{-x} = {1 \over \sqrt{2}}(\ket{+z}-\ket{-z}),$$
%% $$\ket{+y} = {1 \over \sqrt{2}}(\ket{+z}+i\ket{-z}), \enskip
%% \ket{-y} = {1 \over \sqrt{2}}(\ket{+z}-i\ket{-z})$$
{\hthree Identities:}
\vskip 3pt
\Dis 5pt
\baselineskip=24pt
\Fm e^{i x} = \cos x + i \sin x, \qquad e^{i \pi} = -1. \Mf
\Fm e^{x} = 1 + {1 \over 1}x + {1 \over 2!}x^2 + {1 \over 3!}x^3 + ... \Mf
\Fm \sin x = {e^{ix} - e^{-ix} \over 2 i},\Mf
\Fm \cos x = {e^{ix} + e^{-ix} \over 2}\Mf
\Fm \tan x = -i {e^{ix} - e^{-ix} \over e^{ix} + e^{-ix}} = -i {e^{2ix} - 1 \over e^{2ix} + 1}\Mf
\EndDis
{\hthree Definitions:}

Self-adjoint $=$ {\bf Hermitian} $=> A=A^\dagger$. Measurables have hermitian operators.

{\bf Unitary} $=> AA^\dagger=A^\dagger A=I$

Switch from bra to ket by taking complex conjugate.

$\delta-\gamma=\pm  {\pi \over 2} $
\medskip

Photon Polarization \& Spin:

$\phi$: angle between cut axis of the polarizer and the corresponding main axis ($\ket{\sl x}$ or $\ket{\sl y}$). \vskip -8pt
$$\eqalign{\ket{\sl x^\prime} &= \cos{\phi}\ket{\sl x} + \sin{\phi}\ket{\sl y}\cr
\ket{\sl y^\prime} &= -\sin{\phi}\ket{\sl x} + \cos{\phi}\ket{\sl y}\cr
\ket{\sl R} &= {1 \over \sqrt{2}}(\ket{\sl x} + i\ket{\sl y})\cr
\ket{\sl L} &= {1 \over \sqrt{2}}(\ket{\sl x} - i\ket{\sl y})\cr
} $$

Transform from $\ket{\sl x^\prime}$-$\ket{\sl y^\prime}$ basis to the $\ket{\sl x}$-$\ket{\sl y}$ basis.\vskip -8pt
$$S = \left(\matrix{\brakettwo{\sl x}{\sl x^\prime} & \brakettwo{\sl x}{\sl y^\prime} \cr
\brakettwo{\sl y}{\sl x^\prime} & \brakettwo{\sl y}{\sl y^\prime}}\right) = \left(\matrix{\cos{\phi} & -\sin{\phi} \cr \sin{\phi} & \cos{\phi}}\right)$$

$$\ket{\sl R^\prime} = e^{-i\phi}\ket{\sl R} = \hat R(\phi{\bf k})\ket{\sl R} = e^{-i\hat J_z\phi/\hbar}\ket{\sl R}$$
$$\hat J_z\ket{\sl R} = \hbar\ket{\sl R}, \quad \hat J_z\ket{\sl L} = -\hbar\ket{\sl L} $$

{\chapter Matrix Stuff}

$$
\eqalign{
{\rm Identity}&\qquad \sum_{i}{}{\ket{u_i}\bra{u_i}} = 1\cr
{\rm Rotation}&\qquad R(\phi {\bf k}) = e^{-iJ_z\phi / \hbar}\cr
}
$$
\vskip 10pt
\centerline{Change-of-Basis Matrix}
\vskip -5pt
$$S_{z \to x} = \left(
\matrix{\brakettwo{+x}{+z} & \brakettwo{+x}{-z} \cr
\brakettwo{-x}{+z} & \brakettwo{-x}{-z}}
\right)$$
\vskip 10pt
Matrix Operators ($S_z$ basis)
$$
\hat P_+ = \left( \matrix{1&0\cr0&0} \right) \qquad \hat P_- = \left( \matrix{0&0\cr0&1} \right)
$$
$$ \hat J_z \mathrel{\mathop{\longrightarrow}_{S_z}}\left(
\matrix{\hbar/2 & 0 \cr 0 & \hbar/2 }\right)$$
$$ \hat J_z \mathrel{\mathop{\longrightarrow}_{S_z}}\left(
\matrix{\hbar/2 & 0 \cr 0 & \hbar/2 }\right)$$
When doing rotations on eigenstates of $\hat J_z$, we get a diagonal rotation matrix.
$$
\hat R(\phi {\bf k}) \mathrel{\mathop{\longrightarrow}_{S_z}} \left(\matrix{
e^{-i\phi/2} & 0 \cr
0 & e^{i\phi/2} \cr
} \right)
$$
Given $\hat A$ in $S_z$ basis, we can apply $\hat A$ on arbitrary state q in basis b by applying.
$$ S\hat A S^\dagger$$
where S is the change-of-basis matrix from b to z

{\chapter Constants }

$h=6.626 * 10^{-27} {\rm erg\ s}= 10^{-34} {\rm J\ s}$\quad
$\hbar = {h \over 2\pi}$

{\chapter New Stuff}

$\hat R (\phi \vec n) = e^{-i {\bf J }\cdot \vec n \phi / \hbar}$

commutator: $[\hat J_x, \hat J_y] \equiv \hat J_x \hat J_y - \hat J_y \hat J_x$. \quad $[\hat J_x, \hat J_y] = i \hbar \hat J_z$

commutation identities: $[\hat A, \hat B \hat C] = \hat B[\hat A, \hat C] + [\hat A, \hat B]\hat C$\quad 
$[\hat A \hat B, \hat C] = \hat A[\hat B, \hat CC + [\hat A, \hat C]\hat B$

{\hthree Eigenvalues, Eigenstates of Angular Momentum}

the operator ${\bf \hat J^2} = {\bf \hat J} \cdot {\bf \hat J} = \hat J_x^2 + \hat J_y^2 + \hat J_z^2$ commutes with each of the generators of rotation $\hat J_x, \hat J_y, \hat J_z$. Because ${\bf \hat J^2}$ commutes with $\hat J_z$, these operators have simultaneous eigenstates. We label the kets $\ket{\lambda, m}$, where
$\bigJ \ket{\lambda, m} = \lambda \hbar^2 \ket{\lambda, m}$,\quad$\hat J_z \ket{\lambda, m} = m \hbar \ket{\lambda, m}$


{\bf Raising and Lowering Operators}

$\hat J_\pm = \hat J_x \pm i \hat J_y$ (non hermitian). \quad
$[\hat J_z, \hat J_\pm] = \pm \hbar \hat J_\pm$\quad
$m^2 \le \lambda$.

we call ``maximum m value'' $j$. we call ``minimum m value'' $j'$. $j - j' = j - (- j) = 2j =$ an integer. Allowed values of j are $j = 0, 1/2, 1, 3/2, 2, \dots$.

(3.61) $\bra{j,m'} \hat J_+ \ket{j,m} = \sqrt{j(j+1)-m(m+1)}\hbar\delta_{m',m+1}$

$\hat S_+ \ket{s,m} = \sqrt{s(s+1)-m(m+1)}\hbar \ket{s,m+1}$

{\bf Magnitude of Angular Momentum} $\sqrt{j(j+1)}\hbar$

$j = 1$ matrix representation of the raising and lowering operators.
$$\hat J_+ \mathrel{\mathop{\longrightarrow}_{\rm J_z\ basis}} \sqrt{2} \hbar \left(\matrix{0&1&0\cr0&0&1\cr0&0&0}\right)$$

{\bf Uncertainty:} $\Delta J_x \Delta J_y \ge {\hbar \over 2}|\left \langle J_z \right \rangle|$

$\hat S_x = {\hat S_+ + \hat S_- \over 2}$\quad$\hat S_y = {\hat S_+ - \hat S_- \over 2i}$

{\chapter Spin-1/2 Eigenvalue Problem}

In $S_z$ basis, we obtain

$\hat S_x \rightarrow {\hbar \over 2} \left(\matrix{0&1\cr 1&0}\right)$\quad
$\hat S_y \rightarrow {\hbar \over 2} \left(\matrix{0&-i\cr i&0} \right)$\quad
$\hat S_z \rightarrow {\hbar \over 2} \left(\matrix{1&0\cr 0&-1} \right)$
These matrices without the constant factors are called {\bf Pauli Spin Matrices}

{\chapter Stern-Gerlach w/ Spin 1}

$$\hat S_y \mathrel{\mathop{\longrightarrow}_{\rm S_z\ basis}} {\hbar \over \sqrt{2}} \left(\matrix{0& -i & 0 \cr i & 0 & -i \cr 0 & i & 0}\right)$$

{\chapter Summary Chap 3}

$[\hat J_x , \hat J_y] = i \hbar \hat J_z$\quad
$[\hat J_y , \hat J_z] = i \hbar \hat J_x$\quad
$[\hat J_z , \hat J_x] = i \hbar \hat J_y$\quad

Eigenstates of $\hat J_n = \bigJ \cdot \bf n$ can be determined by $\hat J_n \ket{j,m}_n = m \hbar \ket{j,m}_n$

When two hermitian operators do not commute, $[\hat A, \hat B] = i C$, there is a fundamental uncertainty relation $\Delta J_x \Delta J_y \ge {\hbar \over 2}|\left \langle J_z \right \rangle|$.

{\chapter Chap 4}

Time evolution operator $\hat U(t)$ -- $\hat U(t) \ket{\psi (0)} = \ket{\psi(t)}$

$\hat U(t)$ is unitary. $\hat U(dt) = 1 - {i \over \hbar}\hat H dt$, where $\hat H$ is the {\bf generator of time translations}

By differential equation, we have $i \hbar {d \over dt}\hat U(t) = \hat H \hat U(t)$

{\bf Schrodinger equation:} $i \hbar {d \over dt} \ket{\psi(t)} = \hat H \ket{\psi(t)}$

If $\hat H$ is time independent, we can obtain $\ket{\psi(t)} = e^{-i \hat H t / \hbar}\ket{\psi(0)}$

$\hat H$ -- {\bf Hamiltonian}, or energy operator.

If $\hat H$ is time independent, we can obtain a closed-form expression for $\hat U$ from a series of infinitesimal time translations: $\hat U(t) = \lim_{N\rightarrow\infty}\left[1-{i \over \hbar}\hat H \left( t \over N\right)\right]^N=e^{-i \hat H t / \hbar}$

$\langle E \rangle = \braketthree{\psi}{\hat H}{\psi}$

Eigenstates of Hamiltonian, which are the energy eigenstates satisfying $\hat H \ket{E} = E\ket{E}$, play a special role. In this case we have $\ket{\psi (t)} = e^{-i E t / \hbar}\ket{E}$ (stationary state)

{\chapter Precession of Spin 1/2 in Mag Field}

Time evolution of the spin-1/2 particle in a constant magnetic field. z-axis in the direction of the magnetic field, and the charge of the particle to be -e.

$$ \hat H = - {\bf \hat \mu} \cdot B = - {gq \over 2 m c}\hat S B = {g e \over 2 m c} \hat S_z B_0 = w_0 \hat S_z,\ w_0 = geB_0 / 2mc$$






%BOXEND
}


%below we populate the pages!

%\newdimen\fullcolheight
%\fullcolheight=10.5 in

\makeandprintthreefullheightcols\bigbox{\vsize}

%\makeandprintthreefullheightcols\bigbox{\vsize}

%\makeandprintthreefullheightcols\bigbox{\vsize}

%\makeandprintthreefullheightcols\bigbox{\vsize}

%\iflong
%\makeandprintfullheightcols\bigbox{10.5 in}
%\else
%\makeandprintequalheightcols\bigbox
%\fi

%\hrule
%\hbox to \hsize{\box\bigtable}

\end
