%TIKZ enables powerful plotting functionality
\input tikz
\usetikzlibrary{intersections,arrows}

% the following two lines only work with pdftex.  set the paper size to letter, as pdftex
% defaults to A4 and this is not what we want here.
\pdfpagewidth 8.5 true in
\pdfpageheight 11 true in

\font\myheader=cmr17

\centerline{\myheader  My \TeX\ Examples and Exercises }
\rightline{Currell Berry}
\rightline{\number\year --\number\month --\number\day}
\smallskip % This puts a little extra space after the title line.

\beginsection 1. Overview  % The next line must be blank!

This document shows examples of various tex features for my future reference.

\beginsection 2. Text Rotation \& Table % The next line must be blank!

\hbox{
\hbox
  {\pdfsave%
   \pdfsetmatrix{0 -1 1 0}%
   \rlap{Hello}%
    \pdfrestore}

\quad

\hbox
  {\pdfsave%
   \pdfsetmatrix{0 -1 1 0}%
   \rlap{There}%
    \pdfrestore}
\quad

\hbox
  {\pdfsave%
   \pdfsetmatrix{0 -1 1 0}%
   \rlap{This}%
    \pdfrestore}

\quad

\hbox
  {\pdfsave%
   \pdfsetmatrix{0 -1 1 0}%
   \rlap{Is}%
    \pdfrestore}

\quad

\hbox
  {\pdfsave%
   \pdfsetmatrix{0 -1 1 0}%
   \rlap{A}%
    \pdfrestore}

\quad

\hbox
  {\pdfsave%
   \pdfsetmatrix{0 -1 1 0}%
   \rlap{Test}%
    \pdfrestore}
  }
\vskip 5pc

\halign{\indent#\hfil&\quad#\hfil&\quad#\hfil&\quad#\hfil\cr
Horizontal lists&Chapter 14&A&D\cr
Vertical lists&Chapter 15&B&E\cr
Math lists&Chapter 17&C&F\cr}


\beginsection 3. Paragraph Styles % The next line must be blank!

This is a normal lorem ipsum style paragraph for typesetting in the near future but I don't think anyone really cares.

This is a normal lorem ipsum style paragraph for typesetting in the near future but I don't think anyone really cares.

{\narrower
This paragraph will have narrower lines than
the surrounding paragraphs do, because it
uses the ``narrower'' feature of plain \TeX.
The former margins will be restored after
this group ends.\par}

This is a normal lorem ipsum style paragraph for typesetting in the near future but I don't think anyone really cares.

{\leftskip 20pt This is a leftskip lorem ipsum style paragraph for typesetting in the near future but I don't think anyone really cares. \smallskip}

%\rightskip 40pt {This is a normal lorem ipsum style paragraph for typesetting in the near future but I don't think anyone really cares.}

%\pdfliteral direct{ .99619 -.08716 .08716 .99619 0 0 cm } This text is rotated back 5 degrees and should be aligned with the first sentence.  

%\begintable
%Ck\#\vt Date\vt Memo\vt Debit\vt Credit\vt Balance\eltt
%245|8--2|Rent|\$ \hfill 250.00||\$ \hfill 436.29\el
%246|8--2|Danson Electric|\$ \hfill 49.28||\$ \hfill 387.01\el
%247|8--5|Jeff’s Grocery|\$ \hfill 35.88||\$ \hfill 351.13\el
%248||Void|||\el
%249|8--10|Danson Times|\$ \hfill 19.00||\$ \hfill 332.13\el
%250|8--14|Pizza Palace|\$ \hfill 9.95||\$ \hfill 322.18\el
%251|8--15|Jones Hardware|\$ \hfill 45.20||\$ \hfill 276.98\el
%252|8--15|Deposit||\$ \hfill 255.81|\$ \hfill 532.79\el
%253|8--21|Account Fee|\$ \hfill .85||\$ \hfill 531.94\el
%254|8--29|Telephone Co.|\$ \hfill 21.19||\$ \hfill 510.75\endtable

\beginsection 4. Plotting Example % The next line must be blank!

\medskip

\centerline {
\tikzpicture [scale=2]
\clip (-0.8,-0.5) rectangle (2.2,1.2);
\draw[step=.5cm,gray,very thin] (-1.4,-1.4) grid (2.0,1.4);
\filldraw[fill=green!20,draw=green!50!black] (0,0) -- (3mm,0mm)
arc [start angle=0, end angle=30, radius=3mm] -- cycle;
\draw[->] (-1.5,0) -- (1.5,0) coordinate (x axis);
\draw[->] (0,-1.5) -- (0,1.5) coordinate (y axis);
\draw (0,0) circle [radius=1cm];
\draw[very thick,red]
(30:1cm) -- node[left=1pt,fill=white] {$\sin \alpha$} (30:1cm |- x axis);
\draw[very thick,blue]
(30:1cm |- x axis) -- node[below=2pt,fill=white] {$\cos \alpha$} (0,0);
\path [name path=upward line] (1,0) -- (1,1);
\path [name path=sloped line] (0,0) -- (30:1.5cm);
\draw [name intersections={of=upward line and sloped line, by=t}]
[very thick,orange] (1,0) -- node [right=1pt,fill=white]
{$\displaystyle \tan \alpha \color{black}=
{{{\color{red}\sin \alpha}}\over{\color{blue}\cos \alpha}}$} (t);
\draw (0,0) -- (t);
\foreach \x/\xtext in {-1, -0.5/-{{1}\over{2}}, 1}
\draw (\x cm,1pt) -- (\x cm,-1pt) node[anchor=north,fill=white] {$\xtext$};
\foreach \y/\ytext in {-1, -0.5/-{{1}\over{2}}, 0.5/{{1}\over{2}}, 1}
\draw (1pt,\y cm) -- (-1pt,\y cm) node[anchor=east,fill=white] {$\ytext$};
\endtikzpicture.
}


\beginsection 5. Text Circle % The next line must be blank!

%\tracingoutput=0
\hbox{Adapted from \TeX book.}
%\tracingoutput=0

%% \hbox to \hsize {
%% \vbox{
%% \noindent
%% hello
%% }
%% \vbox{
%% \noindent
%% earthlings
%% }
%% \par
%% }

%\newdimen\varunit
%\varunit=1pt % getting ready to make circular insert
% \varunit=1.078pt was used with amr5: it had more letterspacing

%CIRCLESTART
%% \def\dbend{{\manual\char127}} % "dangerous bend" sign
%% \def\d@nger{\medbreak\begingroup\clubpenalty=10000
%%     \def\par{\endgraf\endgroup\medbreak} \noindent\hang\hangafter=-2
%%     \hbox to0pt{\hskip-\hangindent\dbend\hfill} }
%% \outer\def\danger{\d@nger}
%% \def\dd@nger{\medbreak\begingroup\clubpenalty=10000
%% \def\par{\endgraf\endgroup\medbreak} \noindent\hang\hangafter=-2
%% \hbox to0pt{\hskip-\hangindent\dbend\kern1pt\dbend\hfill} }
%% \outer\def\ddanger{\dd@nger}
%% \def\enddanger{\endgraf\endgroup} % omits the \medbreak
\hbox to \hsize {\vtop{\parshape 16
0pc \hsize
0pc \hsize
0pc 29.69pc
0pc 28.51pc
0pc 27.73pc
0pc 27.20pc
0pc 26.85pc
0pc 26.65pc
0pc 26.58pc
0pc 26.65pc
0pc 26.85pc
0pc 27.20pc
0pc 27.73pc
0pc 28.51pc
0pc 29.69pc
0pc \hsize\noindent Lorem ipsum dolor sit amet, consectetur adipiscing elit. Quisque a libero vitae dui ultricies consectetur non nec lacus. Etiam mollis ultrices cursus. Sed blandit tristique consectetur. Vestibulum ante ipsum primis in faucibus orci luctus et ultrices posuere cubilia Curae; Suspendisse non dui ut tortor ultricies tempus ut volutpat felis. Vestibulum ante ipsum primis in faucibus orci luctus et ultrices posuere cubilia Curae; Ut maximus augue in erat pharetra viverra. Aenean vehicula tellus sed orci accumsan porta. Vestibulum auctor eros at metus pulvinar, sed dignissim turpis lacinia. Nunc sit amet urna nisi. Nam fringilla ligula in turpis ultrices semper. Class aptent taciti sociosqu ad litora torquent per conubia nostra, per inceptos himenaeos. Nunc ornare metus ac turpis tincidunt pulvinar. Etiam consectetur tempus nisi semper hendrerit. Sed dignissim vestibulum faucibus. Quisque mollis, ipsum eu dictum lacinia, magna nulla iaculis metus, non posuere nisl nisi accumsan metus. Aliquam erat volutpat. Phasellus aliquet ullamcorper orci, sit amet ultrices enim. Proin massa nulla, porta vel sem in, pellentesque semper urna. Nullam posuere pulvinar dignissim. Quisque porttitor scelerisque massa, in tristique nulla. Maecenas felis purus, ultrices vel egestas eu, sollicitudin ut ante. Maecenas faucibus ut lorem quis venenatis. Nulla facilisi. Duis nec nisi vitae augue finibus molestie sed a urna. Proin tincidunt quis mi accumsan lacinia. 
}
\kern-6pc
\lower 3.2pc
\vtop{
\baselineskip6pt
\parfillskip0pt
\parshape 19
-18.25 pt 36.50 pt
-30.74 pt 61.48 pt
-38.54 pt 77.07 pt
-44.19 pt 88.39 pt
-48.47 pt 96.93 pt
-51.70 pt 103.40 pt
-54.08 pt 108.17 pt
-55.72 pt 111.45 pt
-56.68 pt 113.37 pt
-57.00 pt 114.00 pt
-56.68 pt 113.37 pt
-55.72 pt 111.45 pt
-54.08 pt 108.17 pt
-51.70 pt 103.40 pt
-48.47 pt 96.93 pt
-44.19 pt 88.39 pt
-38.54 pt 77.07 pt
-30.74 pt 61.48 pt
-18.25 pt 36.50 pt
\fiverm
\frenchspacing
\noindent
\hbadness 6000
\tolerance 9999
\pretolerance 0
\hyphenation{iso-peri-met-ric}
The area of a circle is a mean proportional
between any two regular and similar polygons of which one
circumscribes it and the other is isoperimetric with it.
In addition, the area of the circle is less than that of any
circumscribed polygon and greater than that of any
isoperimetric polygon. And further, of these
circumscribed polygons, the one that has the greater number of sides
has a smaller area than the one that has a lesser number;
but, on the other hand, the isoperimetric polygon that
has the greater number of sides is the larger.
\hbox to 36.50pt{\hss[Galileo,\thinspace1638]\hss}
}
}

\beginsection 6. Proclaim and More % The next line must be blank!

\proclaim Theorem 1. \TeX\ has a powerful macro capability.\par

\beginsection 7. Two Columns 

\splittopskip=18.3pt
\def\strutA#1#2{\vrule height#1 depth#2 width0pt}

\setbox1=\vbox{\hsize=3.0in\strutA{\splittopskip}{0pt} Lorem ipsum dolor sit amet, consectetur adipiscing elit. Quisque a libero vitae dui ultricies consectetur non nec lacus. Etiam mollis ultrices cursus. Sed blandit tristique consectetur. Vestibulum ante ipsum primis in faucibus orci luctus et ultrices posuere cubilia Curae; Suspendisse non dui ut tortor ultricies tempus ut volutpat felis. Vestibulum ante ipsum primis in faucibus orci luctus et ultrices posuere cubilia Curae; Ut maximus augue in erat pharetra viverra. Aenean vehicula tellus sed orci accumsan porta. Vestibulum auctor eros at metus pulvinar, sed dignissim turpis lacinia. Nunc sit amet urna nisi. Nam fringilla ligula in turpis ultrices semper. Class aptent taciti sociosqu ad litora torquent per conubia nostra, per inceptos himenaeos. Nunc ornare metus ac turpis tincidunt pulvinar. Etiam consectetur tempus nisi semper hendrerit. Sed dignissim vestibulum faucibus. Quisque mollis, ipsum eu dictum lacinia, magna nulla iaculis metus, non posuere nisl nisi accumsan metus. Aliquam erat volutpat. Phasellus aliquet ullamcorper orci, sit amet ultrices enim. Proin massa nulla, porta vel sem in, pellentesque semper urna. Nullam posuere pulvinar dignissim. Quisque porttitor scelerisque massa, in tristique nulla. Maecenas felis purus, ultrices vel egestas eu, sollicitudin ut ante. Maecenas faucibus ut lorem quis venenatis. Nulla facilisi. Duis nec nisi vitae augue finibus molestie sed a urna. Proin tincidunt quis mi accumsan lacinia.}
\dimen0=\ht1 % The \adjdemerits reference page holds the definition of \tstory
\advance\dimen0 by \dp1
\advance\dimen0 by \splittopskip
\divide\dimen0 by 2
\setbox0=\vsplit1 to \dimen0
\setbox2=\vbox to \dimen0{\unvbox0\vfill}
\setbox3=\vbox to \dimen0{\unvbox1\vfill}

\hrule
\hbox to \hsize{\box2\hfil\box3}
\hrule

\beginsection 8. Bibliography\par % `\par' acts like a blank line.
\frenchspacing % (Chapter 12 recommends this for bibliographies.)
\item{[1]} D.~E. Knuth and M.~F. Plass, ``Breaking paragraphs
into lines,'' {\sl Softw. pract. exp. \bf11} (1981), 1119--1184.
\bye % This is the way the file ends, not with a \bang but a \bye.
\end
