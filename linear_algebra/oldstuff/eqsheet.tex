\documentclass[a4paper,12pt]{article}
\usepackage{graphicx}
\usepackage{amsmath}
\usepackage{braket}
\usepackage[cm]{fullpage}
\usepackage{fancyhdr}
\usepackage[compact]{titlesec}
\usepackage{tabularx}
\usepackage{booktabs}
\usepackage{multirow}
\pagestyle{fancy}
\fancyhead{}
\fancyfoot{}
\fancyhead[RO,RE]{--Currell Berry--}
\fancyfoot[C]{-- Updated 9-24-2014 --}
\fancyfoot[RO, LE] {\thepage}
\newcounter{ctEqn}
\setcounter{ctEqn}{0}

%macro below: first argument: caption.  second argument: equation
\newcommand{\eqentry}[2]{
\setlength{\abovedisplayskip}{0pt}%
\setlength{\belowdisplayskip}{0pt}%
\addtocounter{ctEqn}{1}%
\begin{flalign*}
(\arabic{ctEqn}) \textbf{ #1} &&  #2%
\end{flalign*}
}

%adds some text on below an overflowing equation without starting a new
%entry of its own
\newcommand{\eqadd}[1]{
\begin{flalign*}
&&  #1
\end{flalign*}
}

\newcommand{\eqadj}[1]{\hfill{#1}}

%arguments: caption and text body
\newcommand{\textentry}[2]{
\addtocounter{ctEqn}{1}%

\noindent (\arabic{ctEqn}) \textbf{#1} #2
}

%argument: header text.  resets my counter, calls subsection
\newcommand{\msubsection}[1]{
\setcounter{ctEqn}{0}%
\subsection*{#1}
}

\newcommand{\textadd}[1]{
\hfill #1%
}

\begin{document}
\section*{\center Quantum Mechanics Equations Sheet}

{\setlength{\belowdisplayskip}{0pt}
\msubsection{Test 1 stuff}
\vspace{-4 pt}
\eqentry{definition of hbar}{\hbar = \frac{h}{2\pi} \mbox{, where h is planck's constant}}
\eqentry{uncertainty}{\sqrt{\langle {S_z}^{2} \rangle - \langle S_z \rangle ^{2}} = \Delta S_z \mbox{, where $\Delta S_z$ is uncertainty}}
\eqentry{Completeness Relation}{ \sum_{i}{}{\ket{u_i}\bra{u_i}} = 1}
In a discrete basis $\{\ket{u_i}\}$, an operator is represented by the numbers
\eqentry{Representations of Operators}{\mbox{to represent operator A in basis $\{\ket{u_i}\}$: }A_{ij} = \Bra{u_i}A\Ket{u_j}}
\eqentry{Definition of Raising Operator}{S_+ = S_x+iS_y}
\eqentry{Definition of Lowering Operator}{S_- = S_x-iS_y}
add or subtract the above two equations from each other to obtain for mulas for $S_x$ and $S_y$ respectively

\eqentry{Rotation Operator}{R(\phi \textbf{k}) = e^{-iJ_z\phi / \hbar}}}
(if you want to rotate in another basis, substitute corresponding $J$ operator)

\eqentry{Components of Raising Operator}{S_+ \Ket{s,m} = \sqrt{s(s+1)-m(m+1)}\hbar \ket{s, m+1}}


\msubsection{Spin $1/2$}
\eqentry{$\ket{+x}$ in terms of z basis}{\ket{+x} = \frac{1}{\sqrt{2}}\ket{+z} + \frac{1}{\sqrt{2}}\ket{-z}}
\eqentry{$\ket{-x}$ in terms of z basis}{\ket{-x} = \frac{1}{\sqrt{2}}\ket{+z} - \frac{1}{\sqrt{2}}\ket{-z}}
\eqentry{$\ket{+y}$ in terms of z basis}{\ket{+y} = \frac{1}{\sqrt{2}}\ket{+z} + \frac{i}{\sqrt{2}}\ket{-z}}
\eqentry{$\ket{-y}$ in terms of z basis}{\ket{-y} = \frac{1}{\sqrt{2}}\ket{+z} + \frac{i}{\sqrt{2}}\ket{-z}}


\msubsection{Photon Stuff}

\eqentry{$\Ket{R}$ in terms of $\Ket{x}$ and $\Ket{y}$}{ \frac{1}{\sqrt{2}} (\Ket{x} + i\Ket{y})}
\eqentry{$\Ket{L}$ in terms of $\Ket{x}$ and $\Ket{y}$}{ \frac{1}{\sqrt{2}} (\Ket{x} - i\Ket{y})}


\end{document}
