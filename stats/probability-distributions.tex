%
% plain-TeX file
%
\ifx\pdfoutput\undefined
\input mbboard
\input mathabx
\fi
%\input typofrmt
%\input typotabl
\useoptions{magstep1,a4,english,preprint}

\title{Probability Distributions}
\author{Anthony Phan, \today}
\maketitle

\overfullrule=0pt
\def\description{\medbreak\bgroup
	\def\item##1{\medbreak\hangindent\parindent\leavevmode
		\hskip-\parindent{{\tt##1}.}\enspace\ignorespaces}%
	\def\subitem##1{\smallbreak\hangindent2\parindent\leavevmode
		{\it##1.}\enspace\ignorespaces}%
	\def\subsubitem##1{\par\hangindent3\parindent\leavevmode
		\hskip\parindent{##1.}\enspace\ignorespaces}%
	\let\itemitem=\subitem}%
\def\enddescription{\egroup\medbreak}%
\def\cs#1{\hbox{\tt\string#1}}%
\def\var#1{\ifmmode#1\else$#1$\fi}% usual variable
\def\vari#1{\ifmmode\hbox{\it#1}\else{\it#1}\fi}% variable in italic
\def\vartype#1{\ifmmode\hbox{\tt#1}\else{\tt#1}\fi}% variable type

\section{Continuous Probability distributions\footnote{{\it See}\/ \cs{probability-distributions.h}}}

\description

\item{lnGamma(\vartype{double} $x$)} Logarithm of the Eulerian Gamma
function, $\ln(\Gamma(x))$.

\item{Gamma(\vartype{double} $x$)} Eulerian Gamma function
$$
	\mathbb R_+^*\longrightarrow[\sqrt
	\pi,+\infty\mathclose[,\qquad x\longmapsto \Gamma(x)
	=\int_0^{+\infty}t^{x-1}\,{\rm e}^{-t}\,{\rm d}t.
$$
Remember that for $n\in\mathbb N$, $n!=\Gamma(n+1)$.

\item{Beta(\vartype{double} $x$, \vartype{double} $y$)} Eulerian Beta function
$$
	(\mathbb R_+^*)^2\longrightarrow\mathbb R_+^*,\qquad
	x\longmapsto {\rm B}(x, y)=\int_{0}^{1}u^{x-1}(1-u)^{y-1}\,{\rm d}u
	={\Gamma(x)\times\Gamma(y)\over\Gamma(x+y)}\,.
$$


% Gamma distributions

\item{gammapdf(\vartype{double} $x$, \vartype{double} $a$, \vartype{double} $\lambda$)} Probability density function of
the Gamma distribution with parameter $a>0$, $\lambda>0$.
$$
	\mathbb R\longrightarrow\mathbb R_+,\qquad x\longmapsto
	\mathbb 1_{\mathbb R_+}(x){\lambda^a\over\Gamma(a)}\, x^{a-1}\,{\rm
	e}^{-\lambda x}.
$$

\item{gammacdf(\vartype{double} $x$, \vartype{double} $a$, \vartype{double} $\lambda$)} Cumulative distribution function
of the Gamma distribution with parameter $a>0$, $\lambda>0$.
$$
	\mathbb R\longrightarrow[0,1\mathclose[,\qquad
	x\longmapsto\mathbb 1_{\mathbb R_+}(x){\lambda^a\over\Gamma(a)}
	\int_0^x t^{a-1}\,{\rm e}^{-\lambda t}\,{\rm d}t.
$$

\item{gammaicdf(\vartype{double} $p$, \vartype{double} $a$, \vartype{double} $\lambda$)} Inverse cumulative
distribution function of the Gamma distribution with parameter $a>0$, $\lambda>0$.
$$
	[0,1\mathclose[\longrightarrow\mathbb R_+,\qquad
	p\longmapsto\mathop{\rm gammaicdf}(p,a,\lambda).
$$
It is set to $0$ for $p<\cs{accuracy}$ and to \cs{infinity} for
$p>1-\cs{accuracy}$.

% N(0,1) distribution

\item{normlimit} Numerical parameter for normal computations: if $X$ is a
random variable with law $\mathcal N(0,1)$, the normal distribution
with mean $0$ and standard deviation $1$, then $\mathbb
P\{X\geq\cs{normlimit}\}=\mathbb
P\{X\leq-\cs{normlimit}\}\approx0$. Its value is (unreasonably) set to
$\cs{normlimit} = 10$ ($\cs{normlimit} = 4$ should be sufficient).

\item{normalpdf(\vartype{double} $x$)} Probability density function of
$\mathcal N(0,1)$.
$$
	\mathbb R\longrightarrow\mathbb R_+,\qquad x\longmapsto{\rm
	e}^{-x^2\!/2}/\sqrt{2\pi}.
$$


\item{normalcdf (\vartype{double} $x$)} Cumulative distribution function of
$\mathcal N(0,1)$.
$$
	\mathbb R\longrightarrow\mathopen]0,\mathclose1[,\qquad
	x\longmapsto\Phi(x)=\int_{-\infty}^x{\rm e}^{-z^2\!/2}\, {{\rm
	d} z\over\sqrt{2\pi}}
	={1\over2}+{1\over\sqrt{2\pi}}\sum_{n=0}^\infty
	{(-1)^n\,x^{2n+1}\over (2n+1)\times n!}\,.
$$
It is computed with its associated power series when
$|x|<\cs{normlimit}$, and set to $0$ or $1$ otherwise.

\item{normalcdf\_(\vartype{double} $x$)} Cumulative distribution function
of $\mathcal N(0,1)$.  It is just another implementation of the
previous function with a Gamma cumulative distribution function since
$$
	\Phi(x)= {1\over2}\,\bigl(\mathop{\rm sgn}(x)
	\times\mathop{\rm gammacdf}(x^2\!/2,1/2)+1,1\bigr),
	\qquad\hbox{for all $x\in\mathbb R$.}
$$

\item{normalicdf(\vartype{double} $p$)} Inverse cumulative
distribution function of $\mathcal N(0,1)$.
$$
	\mathopen]0,1\mathclose[\longrightarrow\mathbb R,\qquad
	p\longmapsto\mathop{\rm
	normalicdf}(p)=\Phi^{-1}(p).
$$
It is set to $\pm\cs{infinity}$ for $p$ outside of
$\mathopen]\cs{accuracy},1-\cs{accuracy}\mathclose[$.
Of course, there is also \cs{normalicdf\_}\dots

\remark
Normal distributions with mean $m$ and standard deviation $\sigma$ are not inplemented since they an be easily derived from the standard normal distribution. For instance, one can set
\verbatim
double gaussianpdf(double x, double m, double sigma){
  return normalpdf((x-m)/sigma)/sigma;}
double gaussiancdf(double x, double m, double sigma){
  return normalcdf((x-m)/sigma);}
double gaussianicdf(double p, double m, double sigma){
  return sigma*normalicdf(p)+m;}
\endverbatim
in order to get the probability, cumulative, inverse cumulative distribution functions of the $\mathcal N(m,\sigma^2)$ distribution with $m\in\mathbb R$ and $\sigma>0$.
\endremark

% $\chi^2$ distributions

\item{chisquarepdf(\vartype{double} $x$, \vartype{double} $\nu$)} Probability density function
of $\chi^2(\nu)$, the chi-square (Pearson) distribution with $\nu>0$
degrees of freedom.
$$
	\mathbb R\longrightarrow\mathbb R_+,\qquad x\longmapsto\mathbb
	1_{\mathbb R_+}(x) \,{x^{\nu/2-1}{\rm e}^{-x/2}\over
	2^{\nu/2}\,\Gamma(\nu/2)}\,.
$$

\item{chisquarecdf(\vartype{double} $x$, \vartype{double} $\nu$)} Cumulative distribution
function of $\chi^2(\nu)$.
$$
	\mathbb R\longrightarrow\mathbb R_+,\qquad x\longmapsto\mathbb
	1_{\mathbb R_+}(x) \int_0^x\,{t^{\nu/2-1}{\rm e}^{-t/2} \,{{\rm
	d}t\over 2^{\nu/2}\,\Gamma(\nu/2)}}\,.
$$

\item{chisquareicdf(\vartype{double} $p$, \vartype{double} $\nu$)} Inverse cumulative
distribution function of $\chi^2(\nu)$.
$$
	[0,1\mathclose[\longrightarrow\mathbb R_+,\qquad
	p\longmapsto\mathop{\rm chisquareicdf}(p,\nu).
$$
It is set to $0$ for $p<\cs{accuracy}$ and to \cs{infinity} for
$p>1-\cs{accuracy}$.

% Beta distributions

\item{betapdf(\vartype{double} $x$, \vartype{double} $a$, \vartype{double} $b$)} Probability density function
of the Beta distribution with parameters $a>�$ and $b>0$.
$$
	\mathbb R\longrightarrow\mathbb R_+,\qquad x\longmapsto\mathbb
	1_{[0,1]}(x)\,x^{a-1}(1-x)^{b-1}\!/{\rm B}(a,b).
$$

\item{betacdf(\vartype{double} $x$, \vartype{double} $a$, \vartype{double} $b$)} Cumulative distribution
function of the Beta distribution with parameters $a>0$ and $b>0$.
$$
	\mathbb R\longrightarrow\mathbb R_+,\qquad x\longmapsto
	\int_0^x\mathbb 1_{[0,1]}(u)\,u^{a-1}(1-u)^{b-1} \,{{\rm
	d}u\over {\rm B}(a,b)}\,.
$$

\item{betaicdf(\vartype{double} $p$, \vartype{double} $a$, \vartype{double} $b$)} Inverse cumulative
distribution function of the Beta distribution with parameters $a>0$
and $b>0$.
$$
	[0,1]\longrightarrow\mathbb R_+,\qquad
	p\longmapsto\mathop{\rm betaicdf}(p,a,b).
$$
It is set to $0$ for $p<\cs{accuracy}$ and to $1$ for
$p>1-\cs{accuracy}$.

% Student's (T) distributions

\item{studentpdf(\vartype{double} $x$, \vartype{double} $\nu$)} Probability density function of
$\mathcal T(\nu)$, the Student distribution with $\nu>0$ degrees of
freedom.
$$
	\mathbb R\longrightarrow\mathbb R_+,\qquad x\longmapsto
  	{1\over\sqrt{\nu}\,{\rm B}(\nu/2,1/2)} \biggl(1+{x^2\over
  	\nu}\biggr)^{\!\!-(\nu+1)/2} .
$$

\item{studentcdf(\vartype{double} $x$, \vartype{double} $\nu$)} Cumulative distribution
function of $\mathcal T(\nu)$.
$$
	\mathbb R\longrightarrow\mathbb R_+,\qquad x\longmapsto
  	\int_{-\infty}^x \biggl(1+{z^2\over \nu}\biggr)^{\!\!-(\nu+1)/2}
  	{{\rm d}z\over\sqrt{\nu}\,{\rm B}(\nu/2,1/2)} .
$$
\item{studenticdf(\vartype{double} $p$, \vartype{double} $\nu$)} Inverse cumulative
distribution function of $\mathcal T(\nu)$.
$$
	\mathopen]0,1\mathclose[\longrightarrow\mathbb R,\qquad
	p\longmapsto\mathop{\rm studenticdf}(p,\nu).
$$
It is set to $\pm\cs{infinity}$ for $p$ outside of
$\mathopen]\cs{accuracy},1-\cs{accuracy}\mathclose[$.

% Fisher's (F) distributions

\item{fisherpdf(\vartype{double} $x$, \vartype{double} $\nu_1$, \vartype{double} $\nu_2$)} Probability density
function of $\mathcal F(\nu_1,\nu_2)$, the Fisher distribution with
$\nu_1>0$ and $\nu_2>0$ degrees of freedom ($\nu_1$ is the numerator degree
of freedom, $\nu_2$ the demominator degree of freedom).
$$
	\mathbb R\longrightarrow\mathbb R_+,\qquad x\longmapsto
	\mathbb 1_{\mathbb R_+}(x)\,{(\nu_1/\nu_2)^{\nu_1/2}\,x^{\nu_1/2-1}
	\over {\rm B}(\nu_1/2,\nu_2/2)(1+x\times \nu_1/\nu_2)^{(\nu_1+\nu_2)/2}}.
$$

\item{fishercdf(\vartype{double} $x$, \vartype{double} $\nu_1$, \vartype{double} $\nu_2$)} Cumulative distribution
function of $\mathcal F(\nu_1,\nu_2)$.
$$
	\mathbb R\longrightarrow\mathbb R_+,\qquad x\longmapsto
	\mathbb 1_{\mathbb R_+}(x)\int_0^x
	{(\nu_1/\nu_2)^{\nu_1/2}\,z^{\nu_1/2-1} \over {\rm B}(\nu_1/2,\nu_2/2)(1+z\times
	\nu_1/\nu_2)^{(\nu_1+\nu_2)/2}} \,{\rm d}z.
$$

\item{fishericdf(\vartype{double} $p$, \vartype{double} $\nu_1$, \vartype{double} $\nu_2$)} Inverse cumulative
distribution function of $\mathcal F(\nu_1, \nu_2)$.
$$
	[0,1\mathclose[\longrightarrow\mathbb R_+,\qquad
	p\longmapsto\mathop{\rm fishericdf}(p,\nu_1,\nu_2).
$$
It is set to $0$ for $p<\cs{accuracy}$ and to \cs{infinity} for
$p>1-\cs{accuracy}$.

\section{Discrete Probability distributions\footnote{{\it See}\/ \cs{probability-distributions.h}}}

About quantiles, please note that $q_p=k+0.5$ when
$F(k)=p$ for $k\in\mathbb N$.

% Poisson's distributions

\item{poissonpdf(\vartype{double} $x$, \vartype{double} $\lambda$)} Probability distribution
function of $\mathcal P(\lambda)$, the Poisson distribution with
parameter $\lambda\geq0$.
$$
	\mathbb R\longrightarrow[0,1],\qquad x\longmapsto\cases{{\rm
	e}^{-\lambda}\,\lambda^x\!/x!& if $x\in\mathbb N$,\cr 0
	&otherwise.}
$$
\item{poissoncdf(\vartype{double} $x$, \vartype{double} $\lambda$)} Cumulative distribution
function of $\mathcal P(\lambda)$.

\item{poissonicdf(\vartype{double} $p$, \vartype{double} $\lambda$)} Inverse cumulative
distribution function of $\mathcal P(\lambda)$.

% Binomial distributions

\item{binomialpdf(\vartype{double} $x$, \vartype{int} $n$, \vartype{double} $\pi$)} Probability distribution
function of $\mathcal B(n,\pi)$, the binomial distribution with
parameters $n\in\mathbb N^*$ and $\pi\in[0,1]$.
$$
	\mathbb R\longrightarrow[0,1],\qquad
	x\longmapsto\cases{C_n^x\,\pi^x(1-\pi)^{n-x}& if
	$x\in\{0,1,\dots,n\}$,\cr 0 &otherwise.}
$$

\item{binomialcdf(\vartype{double} $x$, \vartype{int} $n$, \vartype{double} $\pi$)} Cumulative distribution
function of $\mathcal B(n,\pi)$.

\item{binomialicdf(\vartype{double} $p$, \vartype{int} $n$, \vartype{double} $\pi$)} Inverse cumulative
distribution function of $\mathcal B(n,\pi)$.

% Geometric distributions

\item{geometricpdf(\vartype{double} $x$, \vartype{double} $\pi$)} Probability distribution
function of $\mathcal G(\pi)$, the geometrical distribution with
parameter $\pi\in[0,1]$. Describe the law of the first success rank in
an infinitely repeated Bernoulli trial with parameter $\pi\in[0,1]$.
Thus, it is given by
$$
	\mathbb R\longrightarrow[0,1], \qquad x\longmapsto
	\cases{\pi(1-\pi)^{x-1}& if $x\in\{1,2,3,\dots\}$,\cr 0 &
	otherwise.}
$$

\item{geometriccdf(\vartype{double} $x$, \vartype{double} $\pi$)} Cumulative distribution
function of $\mathcal G(\pi)$. It returns the sum up to $x$ of the
previous probabilities. Thus it is given by
$$
	\mathbb R\longrightarrow[0,1], \qquad x\longmapsto
	\cases{1-(1-\pi)^{\mathop{\rm floor}x}& if $x\geq1$,\cr 0 &
	otherwise.}
$$

\item{geometricicdf(\vartype{double} $p$, \vartype{double} $\pi$)} Inverse cumulative
distribution function of $\mathcal G(\pi)$.
$$
	[0,1]\longrightarrow\{1,1.5,2,2.5,3,3.5,\dots\}, \qquad
	p\longmapsto\mathop{\rm geometricicdf}(p,\pi).
$$

\item{negativebinomialpdf(\vartype{double} $x$, \vartype{int} $n$, \vartype{double} $\pi$)} Probability
distribution function of the negative binomial ditribution with
parameters $n\in\mathbb N^*$ and $\pi\in[0,1]$. Describe the law of the
$n$-th success rank, $n\in\mathbb N^*$, in an infinitely repeated
Bernoulli trial with parameter $\pi\in[0,1]$.  Thus, it is given by
$$
	\mathbb R\longrightarrow[0,1],\qquad x\longmapsto
	\cases{C_{x-1}^{n-1}\,\pi^n(1-\pi)^{x-n}& if
	$x\in\{n,n+1,n+2,\dots\}$,\cr 0 & otherwise.}
$$
One can remark that the negative binomial distributions for $n=1$ are
just the corresponding geometric ones. It is certainly better when
$n=1$ to use functions related to geometric distributions instead of
negative binomial distributions related ones.

\item{negativebinomialcdf(\vartype{double} $x$, \vartype{int} $n$, \vartype{double} $\pi$)} Cumulative
distribution function of the negative binomial ditribution with
parameters $n\in\mathbb N^*$ and $\pi\in[0,1]$. One can easily prove
that it is given by
$$
	\mathbb R\longrightarrow[0,1], \qquad x\longmapsto
	1-\mathop{\rm binomialcdf}(n-1,\mathop{\rm floor} x,\pi).
$$

\item{negativebinomialicdf(\vartype{double} $p$, \vartype{int} $n$, \vartype{double} $\pi$)} Inverse
cumulative distribution function of the negative binomial ditribution
with parameters $n\in\mathbb N^*$ and $\pi\in[0,1]$.
$$
	[0,1]\longrightarrow\{n,n+0.5,n+1,n+1.5,\dots\}, \qquad
	p\longmapsto\mathop{\rm
	negativebinomialicdf}(p,n,\pi).
$$

% Hypergeometric distributions

\item{hypergeometricpdf(\vartype{double} $x$, \vartype{int} $N$, \vartype{int} $n$, \vartype{double} $\pi$)} Probability
distribution function of $\mathcal H(N, n,\pi)$, the
hypergeometrical distribution with parameters $N\geq n\in\mathbb N^*$
and $\pi\in[0,1]$.  One should have $N\pi\in\mathbb N$.
$$
\eqalign{
	\mathbb R&\longrightarrow[0,1],\cr
	x&\longmapsto\cases{\displaystyle
	{C_{N\pi}^x\,C_{N(1-\pi)}^{n-x}\over C_N^n}& if
	$x\in\{\max(0,n-N(1-\pi)),\dots,\min(n,N\pi)\}$,\cr 0 &otherwise.}
	}
$$

\item{hypergeometriccdf(\vartype{double} $x$, \vartype{int} $N$, \vartype{int} $n$, \vartype{double} $\pi$)} Cumulative
distribution function of $\mathcal H(N, n,\pi)$.

\item{hypergeometricicdf(\vartype{double} $p$, \vartype{int} $N$, \vartype{int} $n$, \vartype{double} $\pi$)} Inverse
cumulative distribution function of $\mathcal H(N,n,\pi)$.

\section{Rather specific Probability distributions\footnote{{\it See}\/ \cs{probability-distributions.h}}}

\item{Kolmogorovcdf(\vartype{double} $x$, \vartype{int} $n$)} Cumulative
distribution function of Kolmogorov distributions.
These are the famous probability distributions involved in
Kolmogorov--Smirnov (two-sided) Goodness of Fit tests
with statistic
$$
	K_n={\|F-F_n\|}_\infty=\sup_{x\in\mathbb R}\bigl|F(x)-F_n(x)\bigr|
	=\max\nolimits_{i = 1}^n \bigl(F(X_{(i)})-(i-1)/n\bigr)
	\vee\bigl(i/n-F(X_{(i)})\bigr).
$$
Their
computation is based on ``Evaluating Kolmogorov's Distribution''
by George Marsa\-glia and Wai Wan Tsang.

\item{kolmogorovicdf(\vartype{double} $p$, \vartype{int} $n$)} Inverse
cumulative distribution function of Kolmogorov distributions.
(Please do not use it since it is based on the bisection
method or dichotomy.)

\item{klmcdf(\vartype{double} $x$, \vartype{int} $n$)}
Cumulative distribution function of the limiting distributions
associated to Kolmogorov distribution by Dudley's asymptotic
formula (1964):
$$
	\lim_{n\to\infty}\mathbb P\bigl\{K_n\leq u/\!\sqrt n\bigr\}
	=1+2\sum_{k=1}^\infty(-1)^k\exp\bigl(-2k^2u^2\bigr),
$$
with some numerical adjustments (Stephens M.A., 1970).

\item{klmicdf(\vartype{double} $p$, \vartype{int} $n$)} Inverse
cumulative distribution function of the previous distributions.


\item{kpmcdf(\vartype{double} $x$, \vartype{int} $n$)}
Cumulative distribution function of the distribution involved in the
one sided  Goodness of Fit tests with statistic
$$
	K^+_n=\sup_{x\in\mathbb R}\bigl(F(x)-F_n(x)\bigr)
	=\max\nolimits_{1\leq i\leq n}
	\Bigl({i\over n}-F\bigl(X_{(i)}\bigr)\Bigr)
$$
or
$$
	K^-_n=\sup_{x\in\mathbb R}\bigl(F_n(x)-F(x)\bigr)
	=\max\nolimits_{1\leq i\leq n}
	\Bigl(F\bigl(X_{(i)}\bigr)-{i-1\over n}\Bigr)
$$
which share the same distribution: for $x\in[0,1]$,
$$
\eqalign{
        \hbox{\tt kpmcdf}(x,n)
        =\Proba\bigl\{K^\pm_n\leq x\bigr\}
	&=x\sum_{0\leq k\leq nx}{n\choose k}(k/n-x)^k(x+1-k/n)^{n-k-1}\cr
        &=1-x\sum_{nx<k\leq n}{n\choose k}(k/n-x)^k(x+1-k/n)^{n-k-1},\cr
}
$$
the second formula being the one used for computations.

\item{kpmicdf(\vartype{double} $p$, \vartype{int} $n$)} Inverse
cumulative distribution function of the previous distributions.

\enddescription

\newpage
\section*{Appendix: tables\footnote{{\it See}\/ \cs{probability-tables.c}}}

\def\truc#1#2{\subsection*{#1}#2\medbreak
\input #1.table\relax\breakupto{0.25\vsize}}

\truc{normalcdf}{This table provides $\Phi(z)$ for $0\leq z<3$,
where $\Phi$ is the cumulative distribution function of te standard
normal distribution.}
\truc{normalicdf}{This table provides $\Phi^{-1}(\alpha)$ for
$0.5\leq\alpha<1$. Note that $\Phi^{-1}(1-\alpha)=-\Phi^{-1}(\alpha)$.}
\truc{normalicdfbis}{This table provides $\Phi^{-1}(1-\alpha/2)$ for
$0<\alpha<1$.}
\truc{normalicdfter}{This table provides $\Phi^{-1}(1-\alpha/2)$ for
very small values of $\alpha$.}
\truc{chisquareicdf}{This table provides $\cs{chisquareicdf}(1-\alpha,\nu)$.}
\truc{studenticdf}{This table provides $\cs{studenticdf}(1-\alpha/2,\nu)$.}
\truc{fishericdf}{This table provides $\cs{fishericdf}(1-\alpha,\nu_1,\nu_2)$
with $\alpha=0.05$. Values are roughly rounded.}
\truc{kolmogorovicdf}{This table provides $\cs{kolmogorovicdf}(1-\alpha,n)$.}
\truc{klmicdf}{This table provides $\cs{klmicdf}(1-\alpha,n)$, the usual asymptotic approximation of the Kolmogorov inverse cumulative distribution function.}

\bye
