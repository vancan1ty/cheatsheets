%pages to come back to
%390

%TIKZ enables powerful plotting functionality
%\input tikz
%\usetikzlibrary{intersections,arrows}
\input ../core/macros.tex

\input opmac
\input chelvet
\input tx-math
\input ctimes
\typosize[9/10.5]
%\usetimes
%\eightpoint
\font\titlefont =phvb at 13 pt
\font\headeronefont =phvb at 12 pt
\font\headertwofont =phvb at 10pt
\font\smallfont=ptmr at 8pt

\def\title#1{{\titlefont \vskip 0.02 in #1 \vskip 0.03 in}}
\def\headerone#1{{\vskip 0.02 in \headeronefont #1 \vskip 0.02 in}}
\def\headertwo#1{{\vskip 0.02 in \headertwofont #1 \vskip 0.02 in}}

% the following two lines only work with pdftex.  set the paper size to letter, as pdftex
% defaults to A4 and this is not what we want here.
\pdfpagewidth 11 true in
\pdfpageheight 8.5 true in
\nopagenumbers

\voffset=-0.75 in
\hoffset=-0.75 in
\hsize=10.6in
\vsize=8.0in
\parindent=0pt

\font\smallfont=cmr7
\sloppy

%\centerline{Currell Berry -- {\bf Formulas} -- \number\year --\number\month --\number\day}
%\smallskip % This puts a little extra space after the title line.

%\splittopskip=18.3pt
\def\strutA#1#2{\vrule height#1 depth#2 width0pt}

\dimen1=3.40in
\raggedright

\newbox\bigbox
\setbox\bigbox=\vbox{\hsize=\dimen1\title{Physics Equations -- Currell Berry}
\headerone{Fundamental Equations}

\def\dtablerulemulti{\noalign{\enskip\hrule\enskip}}
\def\dtablerulesingle{\noalign{\hrule\enskip}}

\dtablestart{2.0 in}{1.2 in}
{\bf Equation} & {\bf Name}\cr\noalign{\hrule\enskip}
$\vec V$ stays constant IFF $\vec F_{\rm net}= \vec 0$ & Newton's first law\cr\dtablerulesingle
$\vec F_{\rm net} = ma$ & Newton's second law\cr\dtablerulemulti
$\vec F_{\rm A\ on\ B} = - \vec F_{\rm B\ on\ A}$ & Newton's third law\cr\dtablerulesingle
$F = G{m_1m_2 \over r^2}$ & Newton's law of gravity\cr\dtablerulemulti
$\oint \vec E \cdot d\vec A = {Q_{\rm in} \over \epsilon_0}$ & Gauss's law\cr\dtablerulesingle
$\oint \vec B \cdot d\vec A = 0$ & Gauss's law for magnetism\cr\dtablerulemulti
$\oint \vec E \cdot d\vec s = -d{\Phi_m \over dt}$ & Faraday's law\cr\dtablerulesingle
$\oint \vec B \cdot d\vec s = \mu_0I_{\rm through}+\epsilon_0\mu_0 d {\Phi_e \over dt}$ & Ampere-Maxwell law\cr\dtablerulemulti
$\vec F = q(\vec E+\vec v \times \vec B)$ & Lorentz force law\cr\dtablerulesingle
$\Delta E_{\rm th} = W+Q$ & First law of thermodynamics\cr\dtablerulemulti
Entropy of an isolated system never decreases. & Second law of thermodynamics\cr\dtablerulemulti
\dtableend

\headerone{Conservation}

Momentum, energy, angular momentum, charge are conserved for an isolated system.  Mass is conserved in normal situations.

\headerone{Various Equations}
\headertwo{Linear Motion}
$ d = v_it+{1 \over 2}at^2$\quad
$ v_f = v_i+at$\quad
$ v_f^2 = v_i^2+2ad$\quad
$ v_f^2 = v_i^2+2ad$\quad
$ K={1 \over 2}mv^2$\quad
$ \vec{p}=m\vec{v}$\quad
$ \Delta K = J_x=\int_{t_i}^{t_f}{F_x(t)\ dt}$\quad
$ \Delta p_s = W=\int_{s_i}^{s_f}{F_s\ ds}$\quad
\headertwo{Springs}
{\bf Hooke's law:} $(F_{sp})_s=-k\Delta s$\quad
$U_s= {1 \over 2}k(\Delta s)^2$

\headertwo{Rotational Motion}
$ \omega_f=\omega_i + \alpha\Delta t$\quad
$ \theta_f=\theta_i + \omega_i\Delta t + {1 \over 2}\alpha(\Delta t)^2$\quad
$ \omega_f^2=\omega_i^2 + 2\alpha\Delta \theta$\quad
$ a_{\rm tangential}=\alpha r$\quad
$ a_{\rm centripital}=v^2/r=\omega^2r$\quad
$ x_{\rm cm} = {1 \over M}\int{x\ dm}$\quad
$ I=\sum_im_ir_i^2$\quad
$ I=\int{r^2 dm}$\quad
$ K_{\rm rot}={1 \over 2}I\omega^2$\quad
$ E_{\rm mech}=K_{\rm rot}+U_g={1 \over 2}I\omega^2+Mgy_{\rm cm}$\quad
{\bf parallel axis theorem:} $ I = I_{\rm cm}+Md^2$\quad
$ \tau \equiv rF\sin{\phi}$\quad
$ \alpha = {\tau_{\rm net}\over I}$\quad
$ v_{\rm cm}=R\omega$\quad
$ K_{\rm rolling}=K_{\rm rot}+K_{\rm cm}$\quad
$ \vec{\tau}=\vec{r}\times\vec{F}$\quad
$ \vec{L}=\vec{r}\times\vec{p}$\quad
$ {d\vec{L}/dt}=\vec{\tau}_{net}$\quad
$ \vec{L}=I\vec{\omega}$\quad

\headertwo{Planets}
$F_{\rm 1 on 2}=F_{\rm 2 on 1}={Gm_1m_2 \over r^2}$\quad
{\bf Satellite Speed:} $ v=\sqrt{GM\over r}$\quad
{\bf Escape Velocity:} $ v=\sqrt{2GM\over r}$\quad
{\bf On Surface:} $ g={GM\over R_L}$\quad
$ U_g={Gm_1m_2 \over r}$\quad
{\bf Kepler's 3rd:} $ T^2=\left(4\pi^2 \over GM \right)r^3$\quad
{\bf Kepler's 2nd:} $ {\Delta A \over\Delta t} = {L \over 2m}$\quad

\headertwo{Simple Harmonic/Circular Motion}

Uniform circular motion projected onto one dimension is simple harmonic motion.

Any system with a linear restoring force will undergo simple harmonic motion around the equilibrium position.
$x(t) = A\cos{(\omega t + \phi_0)}$\quad
$v_x(t) = -\omega A\sin{(\omega t + \phi_0)}$\quad
{\bf pendulum:} $ \omega = 2\pi f = \sqrt{g \over L}$\quad
{\bf damped oscillator:} $ x(t) = Ae^{-bt/2m}\cos(\omega t + \phi_0)$\quad
{\bf time constant:} $ \tau = m/b$\quad
{\bf damped system:} $ E = E_0e^{-t/\tau}$\quad

\headertwo{Fluids and Elasticity}

Archimedes' principle: The magnitude of the buoyant force equals the weight of the fluid displaced by the object.

Ideal-fluid model: Incompressible. Smooth, laminar flow. Nonviscuous.

Bernoulli's is a statement of energy conservation.
%TODO come back and fix up elasticity stuff
$ p = F/A$\quad
$ p_g = p-1$\quad
$ \rho = m/V $\quad
$ v_1A_1=v_2A_2$\quad
{\bf Bernoulli's:} $ p_1+{1 \over 2}\rho g y_1 = p_2 + {1 \over 2}\rho v_2^2 + \rho gy_2$\quad
$ (F/A) = Y(\Delta L/L)$\quad
$ p = -B(\Delta V/V)$\quad

\headertwo{Matter}

Phases: solid, liquid gas. Ideal-gas model. Isochoric process $\rightarrow$ $V$ constant and $W$=0, Isobaric $\rightarrow$ $p$=constant, Isothermal $\rightarrow$ $T$ constant and $\Delta E_{th}=0$, Adiabatic $\rightarrow$ $Q$=0. conduction, convection, radiation, evaporation. 

%CB TODO come back and figure out monatomic, diatomic gases, elemental solids.
%CB TODO ideal gas summary?
{\bf Second law:} entropy cannot decrease.
{\bf Ideal Gas Law:} $ pV = nRT $\quad
{\bf First Law of Thermo:} $ \Delta E_{th}= W + Q $\quad
$ W = -\int_{V_{\rm i}}^{V_{\rm f}}{p\ dV} $\quad
{\bf specific heat:} $ Q = Mc\Delta T $\quad
$ \epsilon_{\rm avg}={3 \over 2}k_{\rm B}T$\quad
$ p = {2 \over 3}{N \over V}\epsilon_{\rm avg}$\quad
${Q \over \Delta t} = e \rho A T^4$\quad

\headertwo{Waves}

Transverse, Longitudinal.  Snapshot graph, history graph. Superposition, nodes, and antinodes.

$ v = \lambda f$\quad
$ \omega = vk$\quad
$ D(x,t) = A\sin{(kx - \omega t + \phi_0)}$\quad
$ I = P/a$\quad
$ I \propto A^2$\quad

{\bf Doppler: }$ f_{\pm} = {f_0 \over 1 \mp v_s/v}$\quad
$ f_{\pm} = 1 \pm {v_o \over v}f_0 $\quad
{\bf Double slit:} angles of bright fringes: $\theta_m=m{\lambda \over d}$ where m = $0,1,2,\dots$, $d$ is slit spacing.
\quad $I_{double}=4I_1\cos^2{{\pi d \over \lambda L}y}$.\quad
{\bf Diffraction grating:} angles of bright fringes: $d\sin{\theta_m} = mA$, $m=0,1,2,\dots$.\quad
{\bf Single slit:} angles of dark fringes $\theta_p=p{\lambda \over a}$, $p = 1,2,3,\dots$\quad
{\bf Circular aperture:} $w = 2y_1=2L\tan{\theta_1}\approx {2.44\lambda L\over D}$\quad
{\bf de broglie wavelength:} $ \lambda = {h \over p} = {h \over mv}$\quad
{\bf wave intensity:} $I = P/A$\quad {\bf radiation pressure} $p_{\rm rad} = F/A = I/c$

\headertwo{Optics}
{\bf Snell's:} $ n_1\sin(\theta_1) = n_2\sin(\theta_2) $\quad

\headertwo{Electricity and Magnetism}
{\bf Coulomb's} $ F_{\rm 1\ on\ 2} = F_{\rm 2\ on\ 1}={K|q_1||q_2| \over r^2}$\quad
$ \vec{E}(x,y,z) = {\vec{F}_{\rm on\ q}{\rm\ at\ } (x,y,z) \over q} $\quad
{\bf point charge:} $\vec{E} = {1 \over 4\pi\epsilon_0}{q \over r^2}\hat{r}$\quad
$ \vec{E}_{\rm dipole} = {1 \over 4\pi\epsilon_0}{2 \vec{p} \over r^3} $ (on axis)\quad
{\bf gauss's law} $\phi_e = \oint \vec{E}\cdot d\vec{A}={Q_{\rm in} \over \epsilon_0}$\quad
{\bf unif. elec. field:} $U_{\rm elec} = U_0 + qEs $\quad
{\bf point charges:} $U_{\rm elec} = {Kq_1q_2 \over r} $\quad
{\bf dipole:} $U_{\rm dipole} = -\vec{p}\cdot\vec{E} $\quad
$ U_{q + {\rm sources}} = qV $\quad

Gauss's Law, Faraday's Law, Ampere-Maxwell law, Lorentz force law (see fundamentals).

\headertwo{Special Relativity}
{\bf proper time} $\Delta \tau$: time in reference frame for which clock is at rest.\quad
{\bf time dilation} $\Delta t = {\Delta \tau \over \sqrt{1-\beta^2}} \ge \Delta \tau$\quad
{\bf proper length} $\ell$: length in reference frame where objects are at rest\quad
{\bf length contraction} $L' = \sqrt{1-\beta^2} \le \ell$\quad
{\bf spacetime interval} $s^2 = c^2(\Delta t)^2-(\Delta x)^2$ (invariant)\quad

We have inertial reference frames $S$ and $S'$. $S'$ is travelling at velocity $v$ along x-axis, relative to $S$. Origins of $S$ and $S'$ coincide at $t=0$.

{\bf Lorentz transformations: }
\dtablestart{1.6 in}{1.6 in}
$x' = \gamma(x - vt) $ & $x = \gamma(x' + vt')$\cr
$y' = y $ & $y = y'$\cr
$z' = z $ & $z = z'$\cr
$t' = \gamma(t - vx/c^2)$&$t = \gamma(t' + vx'/c^2)$\cr
$u' = {u-v \over 1 - uv/c^2}$&$u = {u' + v \over 1 + u'v/c^2}$\cr
\dtableend

Where $u,u'$ are the $x,x'$ components of an object's velocity, $\beta = {v \over c}$, and $\gamma = 1/\sqrt{1-v^2/c^2}=1/\sqrt{1-\beta^2}$

\headertwo{Emission Spectra and Atomic Physics}

{\bf energy of electromagnetic radiation:} $E = hf$\quad
$p = \gamma_p m u$\quad
{\bf Photoelectric effect} Light can eject electrons from a metal only if $f \ge f_0 = E_0 / h$, where $E_0$ is the metal's {\it work function}. {\bf stopping potential} $V_{\rm stop} = {h \over e}(f-f_0)$
{\bf Wien's law} (blackbody radiation) $\lambda_{\rm peak} = {2.90 \times 10^6 {\rm nm\ K} \over T}$\quad
{\bf Particle in a box} $E_n = {h^2 \over 8 m L^2}n^2, n = 1,2,3,\dots$\quad
{\bf mass-energy equiv:} $E = mc^2 + (\gamma_p - 1)mc^2=\gamma_p m c^2$,
$\gamma_p = {1 \over \sqrt{1-u^2/c^2}}$\quad
{\bf Bohr Model (for hydrogen):} $r_n = n^2 a_B$\quad $E_n = - {1 \over n^2} {e^2 \over 4 \pi \epsilon_0 2 a_B} = - {13.5 {\rm eV} \over n^2}$, $a_B = {4 \pi \epsilon_0 \hbar^2 \over m e^2} \approx 0.0529 nm$\quad

\headertwo{Quantum}
{\bf wave function} $\psi(x)$ determines the probability that a particle will be found in some region of space.\quad
Prob(in $\delta x$ at $x$) = $|\psi(x)|^2 \delta x$

{\bf Heisenberg Uncertainty:} $\Delta x \Delta p_x \ge h/2$\quad {\bf wave packet} $\Delta f \Delta t \approx 1$.\quad
{\bf Schrodinger Eq.} ${d^2\psi \over d x^2} = - {2 m \over \hbar^2} [E - U(x)] \psi(x)$



}


\newbox\zbox

\setbox\zbox=\vbox{\hsize=\dimen1
\headerone{Constants, Symbols, Units}
{\bf Base SI Units} length:~m, mass:~kg, time:~s, current:~A(ampere), temp:~K, amount:~mol, luminous intensity: cd(candela)

{\bf Selected Other Units} $1 {\rm\ eV} = 1.602 \times 10^{-19} $ joules.

\headertwo{Symbols}
\Hrule
\halign{$#$\hfil&\enskip#\hfil&\vtop{\parindent=0pt\hsize=2.0in\strut#\strut}\cr
   & \bf Name & \bf Units \cr
A & area & $\rm m^2$ \cr
    & amplitude & \cr
a & acceleration & $\rm m/s^2$ \cr
& area & \cr
\vec{B} & magnetic field 1 & $\rm 1\  tesla = 1\ T \equiv 1\ N/A\ m$ (flux density)\cr
b & damping constant & kg/s\cr
C & capacitance & $\rm 1\ farad = 1\ F \equiv 1\ C/V$\cr
c & speed of light & 299,792,458\ m/s \cr
  & specific heat & J / kg K \cr
d & distance & m \cr
\vec{E} & electric field & 1\ N/C = 1\ V/m \cr
E & energy & 1\ {\rm joule} = 1\ J = $\rm 1\ kg\ m^2 / s^2$\cr
e & various & 2.71828\dots, electron, elem. charge.\cr
 & electron & \cr
 & elem. charge & $1.60 \times 10^{-19}$\cr
& various & 2.71828\dots, electron, elem. charge.\cr
F & force & 1\ N = $\rm 1\ kg\ m / s^2$ \cr
f & frequency & frequency (1  Hz = 1/s)\cr
  & various & function, friction (N) \cr
G & gravity constant & $\rm 6.674*10^{-11}\ N\ m^2/kg^2$ \cr
g & accel. d.t. gravity & $\rm m / s^2$ \cr
\vec{H} & magnetic field 2 & A/m (field strength)\cr
h & height & m \cr
h & planck's constant & $\rm 6.626*10^{-34}\ J\ s$\cr
\hbar & reduced planck's & $\rm h / 2\pi$\cr
I & intensity & $\rm W / m^2$ \cr
  & electric current & 1 ampere = 1 A = 1 C/s \cr
  & mmnt. of inertia & $\rm kg\ m^2 $ -- ``rotational mass'' \cr
i & imaginary unit & $\sqrt{-1}$ \cr
\bf\hat i & x-axis unit vec & also $\bf\hat j$, $\bf\hat k$ for y and z axes\cr
J & impulse & kg\ m/s  -- equiv to $\Delta P$\cr
K & kinetic energy & J\cr
& electrostatic c. & $8.99 \times 10^9 {\rm\ N\ m^2/C^2}$\cr
k_{\rm B} & boltzmann const. & ${\rm 1.381*10^{-23}\ J/K} = R/N_A$\cr
   & wave number & ${\rm rad}/m$ -- ``spacial freq. of wave''\cr
   & spring constant & J/$\rm m^2$\cr
L & inductance & $\rm 1\ henry=1\ H  \equiv 1\ Wb/A = 1\ T\ m^2/A$\cr
  & ang. momentum & $\rm kg\ m^2/s$\cr
l & length & $\rm m$\cr
m & mass & kg\cr
N & various & normal vector, atomic number\cr
N_{\rm A} & avogadro's num & $6.02*10^{23}$ 1/mol\cr
n & ind. of refraction & unitless -- n = c/v \cr
 & quantum number & $n=1,2,3,\dots$, parameterizes quantum energy state for particle\cr
\vec{p} & momentum & kg m/s -- $\vec{p} \equiv m\vec{v}$ \cr
p & pressure & 1 pascal = 1 Pa $\equiv$ $\rm 1\ N/m^2$ \cr
\vec{p} & dipole moment & $qs$, from the negative to the positive charge\cr
Q & heat & 1 joule = 1 J = 0.2389 cal \cr
q & elect. charge & 1 coulumb = 1 C = 1 A s -- ($q$ or $Q$)\cr
R & elect. resistance & 1 ohm = 1 $\Omega$ = 1 V/A\cr
 & gas constant & 8.314 J/mol K\cr
r & radius & m\cr
S & entropy & \cr
S & entropy & \cr
s & arc length & m\cr
  & position & m\cr
T & period & s\cr
  & abs. temperature & 1 kelvin = 1 K = $T_C$+273\cr
t & time & s\cr
U & potential energy & 1 joule = 1 J\cr
$u$ & atomic mass unit & 1 u = $1.66 * 10^{-27}$ kg\cr
V & elec. potential & 1 volt = 1 V = 1 J/C\cr
  & volume & $\rm m^3$\cr
\vec{v} & velocity & m/s\cr
W & work & 1 N m = 1 kg $\rm m^2/s^2$ = 1 J\cr
w & width & m \cr
x & displacement & m \cr
Z & elec. impedance & 1 ohm = 1 $\Omega$ \cr
\alpha & ang. accel & rad/$s^2$\cr
\Delta & change in var. & used to signify change i.e. $\Delta x$\cr
\epsilon & permittivity & F/m = $\epsilon_r\epsilon_0$\cr
\epsilon_0 & vac. permittivity & $8.854*10^{-12}$ F/m\cr
\theta & angle & rad \cr
\lambda & wavelength & m \cr
\mu & mag. moment & A $m^2$ \cr
\mu & coeff. friction & unitless\cr
\mu & permeability & H/m = $\rm N/A^2$ -- $\mu = \mu_0\mu_r$\cr
\mu_0 & perm const. & $4\pi * 10^-7 {\rm T\ m/A}$\cr
\pi & $\pi$ & 3.14159\dots\cr
\rho & mass density & $\rm kg/m^3$ --- $\rho = m/V$\cr
& resistivity &$\rm \Omega\ m$ --- $\rho = 1/\sigma$\cr
\sigma & conductivity & 1/$\Omega$ m\cr
&stefan-b. constant&$5.67\times 10^{-8}\ W/m^2K^4$\cr
\tau & torque & N m --- $\tau = \vec{r}\times\vec{F}$\cr
 & time constant & different for circuits, oscillations, etc.\cr
 & 2$\pi$& 6.28319\dots\cr
\Phi & field strength & units vary dep. on context\cr
\Phi_e & electric flux & $\int_{\rm surface}\vec{E}\cdot d\vec{A}$\cr
\Phi_m & magnetic flux & 1 weber = 1 Wb = 1 T $m^2$\cr
\phi & phase & radians --- operand to sinusoidal fn.\cr
\psi & wave function & unitless, represents q.m. state\cr
\Omega & elec. resistance & 1 ohm = 1 $\Omega$ = 1 V/A\cr
\omega & ang. velocity & rad/s\cr
}

$F \rightarrow N = {kg\ m \over s^2}$
\headertwo{Miscellaneous}

$\vec{A}\times\vec{B} \equiv AB\sin \alpha$, in the direction given by right-hand rule
}

%below we populate the pages!

\makeandprintthreefullheightcols\bigbox{\vsize}

\makeandprintthreefullheightcols\zbox{\vsize}

%{\smallfont Currell Berry, 2016}
%\medskip
%\hrule

\end
